% !TEX root = main.tex

We needed to have an exact solution in order to test to see if our code was even slightly working. This equation is for the radially symmetric $P_1$ case. 

\begin{align*}
\begin{bmatrix}
p \\
u_r
\end{bmatrix} _{,t}
+
\begin{bmatrix}
0 & \frac{1}{\sqrt{3}} \\
\frac{1}{\sqrt{3}} & 0
\end{bmatrix}
\begin{bmatrix}
p \\
u_r
\end{bmatrix} _{,r}
=
\begin{bmatrix}
-\frac{1}{r\sqrt{3}} u_r \\
-\sigma u_r
\end{bmatrix}
\end{align*}

Where $r \in [-1, 1]$ and r is on the Chebyshev points. 

We can decouple this system.

\begin{align*}
&\vec{w} = \mat{R^{-1}}
\begin{bmatrix}
p \\
u_r
\end{bmatrix} \\
&w_1 = \frac{1}{2}(p - u_r), \quad
w_2 = \frac{1}{2}(p + u_r) \\
&\text{Alternatively} \\
&p = w_1 + w_2, \quad
u_r = w_2 - w_1
\end{align*}

This becomes:

\begin{align*}
\begin{bmatrix}
w_1 \\
w_2
\end{bmatrix} _{,t}
+ 
\begin{bmatrix}
-\frac{1}{\sqrt{3}} & 0 \\
0 & \frac{1}{\sqrt{3}}
\end{bmatrix}
\begin{bmatrix}
w_1 \\
w_2
\end{bmatrix} _{,r}
=
\begin{bmatrix}
\frac{1}{2r\sqrt{3}} - \frac{\sigma}{2} & -\frac{1}{2r\sqrt{3}} + \frac{\sigma}{2} \\
\frac{1}{2r\sqrt{3}} + \frac{\sigma}{2} & -\frac{1}{2r\sqrt{3}} - \frac{\sigma}{2}
\end{bmatrix}
\begin{bmatrix}
w_1 \\
w_2
\end{bmatrix}
\end{align*}
Which is equivalent to:
\begin{align*}
\vec{w}_{,t} + \mat{\Lambda} \, \vec{w}_{,r} = \mat{F(r)} \, \vec{w}
\end{align*}
Note: The $\mathcal{W}$ is the discretized form over the mesh.
\begin{align*}
\vec{\mathcal{W}_1}_{,t} - \frac{1}{\sqrt{3}} \mat{D} \, \vec{\mathcal{W}_1} = \text{diag}(F_{11}) \vec{\mathcal{W}_1} + \text{diag}(F_{12})\vec{\mathcal{W}_2} \\
\vec{\mathcal{W}_2}_{,t} - \frac{1}{\sqrt{3}} \mat{D} \, \vec{\mathcal{W}_2} = \text{diag}(F_{21}) \vec{\mathcal{W}_1} + \text{diag}(F_{22})\vec{\mathcal{W}_2}
\end{align*}
Which is equivalent to:
\begin{align*}
\begin{bmatrix}
\vec{\mathcal{W}_1} \\
\vec{\mathcal{W}_2}
\end{bmatrix} _{,t}
=
\begin{bmatrix}
\frac{1}{\sqrt{3}} \mat{D} + \text{diag}(F_{11}) & \text{diag}(F_{12}) \\
\text{diag}(F_{21}) & -\frac{1}{\sqrt{3}} \mat{D} + \text{diag}(F_{22})
\end{bmatrix}
\begin{bmatrix}
\vec{\mathcal{W}_1} \\
\vec{\mathcal{W}_2}
\end{bmatrix} 
\end{align*}