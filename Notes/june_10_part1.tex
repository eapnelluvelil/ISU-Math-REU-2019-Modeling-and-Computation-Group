% !TEX root = main.tex

We began by setting up the matrix coefficients that represent the system of equations described by (8) and (9) in the Brunner, Holloway paper, reproduced below.

The following represents the three-dimensional Boltzmann equation, for $0 \leq l < \infty$ and $-l \leq m \leq l$.
and for $m \neq 0$:
\begin{align*}
	\frac{\partial}{\partial t}\psi_{l}^{m} &+ \frac{1}{2}\frac{\partial}{\partial x}
	(
	-C_{l-1}^{m-1}\psi_{l-1}^{m-1} + 
	D_{l+1}^{m-1}\psi_{l+1}^{m-1} + 
	E_{l-1}^{m+1}\psi_{l-1}^{m+1} + 
	-F_{l+1}^{m+1}\psi_{l+1}^{m+1} 
	)\\ &+ 
	\frac{\partial}{\partial z}
	(
	A_{l-1}^{m}\psi_{l-1}^{m} + 
	B_{l+1}^{m}\psi_{l+1}^{m}
	) + \Sigma_t\psi^{m}_{l} = 0.
\end{align*}

For $m = 0$:
\begin{align*}
	\frac{\partial}{\partial t}\psi_{l}^0 + 
	\frac{\partial}{\partial x}
	(
	E_{l-1}^{1}\psi_{l-1}^{1} + 
	-F_{l+1}^{1}\psi_{l+1}^{1} 
	) +  
	\frac{\partial}{\partial z}
	(
	A_{l-1}^{0}\psi_{l-1}^{0} + 
	B_{l+1}^{0}\psi_{l+1}^{0}
	) + \Sigma_t\psi^0_{l} = \Sigma_s\psi^0_0\delta_{l0}
\end{align*}

(The specific values for the coefficients $A_l^m$, $B_l^m$, etc. can be found in the paper by Brunner and Holloway)

Additionally, the identity $\bar{\psi_l^m} = (-1)^m\psi_l^{-m}$ takes advantage of each $\psi$ being real to remove terms with negative $m$.
Thus $m$ goes from $0$ to $l$ instead of $-l$ to $l$.
Under the $P_N$ approximation, it is assumed that $\psi_l^m = 0$ whenever $l > m$, $l < 0$, or $l > m$.
This results in the system of equations represented by the matrices produced by $\texttt{pn\_approx.py}$, which produces the matrices $A, B$, and $C$ in the following system:
\begin{align*}
	\vec{q}_{,t} + \mat{A}\,\vec{q}_{,x} + \mat{B}\,\vec{q}_{,z} = \mat{C}\,\vec{q}
\end{align*}
Unlike the paper by Brunner and Holloway, the functions $\psi_l^m$ are grouped by $l$ rather than $m$.
As a result, the following mapping from $(m, l)$ to equation number $p$ is used:
\begin{align*}
	p = \frac{l(l+1)}{2} + 1 + m
\end{align*}
This mapping causes the exact matrices produced in the paper by Brunner and Holloway and our own to be distinct, but only as permutations of one another.
