% !TEX root = main.tex

In our code, named $\texttt{pn\_approx.py}$, we set functions for A, B, C, D, E, F that can be found in the Brunner Holloway paper on two-dimensional time dependent Riemann solvers for neutron transport in equations (4), (5), and (6). Then, we defined a python function, $\texttt{pn\_matrices}$, to loop through different values of $\ell$ and $m$ to create our matrices. In order for this to work correctly, we needed to have if statements for each type of value. This could be improved upon at a later time, but for now, it allows our code to run without causing issues with our $\ell$ and $m$ values to be less than $0$ or greater than $N$. We also imported the sys library of python in order to run different $N$ values from our terminal instead of having to change our code each time we tested a different $N$ value. \\

Note: $N$ is always a positive, odd integer. \\

We needed to only deal with one variable at a time instead of dealing with $m$ and $\ell$ at the same time. The following equation outlines a way that we were able to do that:

\begin{align*}
p = &\frac{\ell(\ell + 1)}{2} + 1 + m \\
\ell = &0, 1, 2, \dots, N \\
m = &0, 1, 2, \dots, \ell \\
1\leq p \leq &\frac{N(N + 1)}{2} + 1 + N \\
1\leq p \leq &\frac{N^2}{2} + \frac{3N}{2} +1
\end{align*}

The following are the two cases for our equations:

\begin{align*}
\text{For } m \neq 0 \text{:} \\
\psi_{p,t} + \frac{1}{2} \frac{\partial}{\partial x} [-C_{\ell - 1}^{m-1} \psi_{p-\ell-1} + D_{\ell + 1}^{m-1} \psi_{p + \ell} + E_{\ell - 1}^{m + 1} \psi_{p- \ell +1} - F_{\ell + 1}^{m+1} \psi_{p + \ell + 2} ] \\ + \frac{\partial}{\partial z} [A_{\ell - 1}^{m} \psi_{p-\ell} + B_{\ell + 1}^{m} \psi_{p+\ell+1}] + \sum_t \psi_p = 0 \\ \\
\text{For } m = 0 \text{:} \\
\psi_p + \frac{\partial}{\partial x}[E_{\ell - 1}^1 \psi_{p-\ell +1} + F_{\ell + 1}^1 \psi_{p+\ell+2}] + \frac{\partial}{\partial z}[A_{\ell - 1}^0 \psi_{p-\ell} + B_{\ell+1}^0 \psi_{p+\ell+1}] + \sum_t \psi_p = \sum_s \psi_1 \delta_{\ell 0}
\end{align*}

The goal is to have the average approach the $\psi_0^0$ value. When our $p = 1$, our average value is $\psi_0^0$. 

\begin{align*}
\psi_{p,t} = -\sum_t \psi_p \\
\psi_{1,t} = -\sum_t \psi_1 + \sum_s \psi_1
\end{align*}

Going forward, we need to make a $\widetilde{\mat{A}} (\omega) = \cos(\omega) \mat{A_x} + \sin(\omega) \mat{A_z}$. We then send $\widetilde{\mat{A}}$ to the code that we have to solve matrices of this form and repeat for multiple values of $\omega$. Then, we need to address the fact that we don't have a uniform mesh because we are using Chebyshev points, so methods like the discrete Radon transform (DRT) won't work the same. \\
The easier part of the Radon transform will be doing the forward Radon transform, since it is just integration. However, the inverse Radon transform will be where we will struggle. 
