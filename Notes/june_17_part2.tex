% !TEX root = main.tex

We make a new file $\texttt{radon\_transform\_v6.py}$ to take the nearest 4 points and interpolate between them to find our point. We copied $\texttt{radon\_transform\_v5.py}$ and then pasted it into $\texttt{radon\_transform\_v6.py}$ to have a base for our code. \\
When we were changing the code, we found our $\omega$ values by subtracting one from the previous $\omega$ lower index and adding one to the previous $\omega$ upper index. Then we did the same thing with our s-values and our function values. Then we had to code our interpolation method. We still had to code the summation from Friday into our interpolation method, so we created variables to do that, called $\texttt{p1, p2, p3, p4}$. We also had to create the angle $\theta_1$, which is the angle halfway between points $p_2$ and $p_3$. We accomplished this by taking the change in $\omega$ and dividing that by 2, then adding our $\omega_2$ to that. Then we plugged all of that information into our interpolation equation.