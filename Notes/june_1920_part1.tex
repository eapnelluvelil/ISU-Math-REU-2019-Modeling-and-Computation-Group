% !TEX root = main.tex

A significant portion of these days were spent making the Radon Transform software more efficient and fast.
This was primarily performed through Numpy vectorization.
Initially, the quadrature used to calculate the Radon Transform for a particular discretization point was vectorized, referred to as "vectorized in $q$".
This resulted in approximately a 10x speedup.
Next, the code was fully vectorized, being vectorized in $s$, $\omega$, and $q$.
For trivially coarse domains, this made the code significantly faster. 
However, for any reasonably sized domain, this vectorization caused the overall calculation to take either more time, or not compute at all due to memory constraints.
The unexpected slow-down in runtime is likely the result of cache issues, but the exact causes were not investigated fully.
To alleviate these issues, we attempted to vectorize the code only in $\omega$ and $q$, having the effect of computing the radon transform for entire diameters of the domain at once.
While this did not have the same memory issues as the original method, it remained only mildly more efficient at best, and slower at worst.
As a result, the version of the software vectorized in $q$ has become the de-facto radon transform to be used in future work.