% !TEX root = main.tex

\section*{Testing iterative methods with the LU decomposition}
We wrote code to solve the modified system that Dr. Rossmanith proposed yesterday.
We first computed $\mat{R}$ using \verb|radon_matrix.py|, then computed $\mat{R^{T} R}$ and its corresponding eigenvalues.
Although $\mat{R^{T} R}$ is a symmetric, semi-positive-definite matrix i.e. it has real and non-negative eigenvalues in exact arithmetic, $\mat{R^{T} R}$ will have eigenvalues with a negligible imaginary component due to floating point calculations.
We then selected the minimum eigenvalue $\lambda_{min}$ and computed the magnitude $mag$ of $\lambda_{min}$ by taking $mag = \log_{10} (\lambda_{min} + \epsilon_{mach})$, where $\epsilon_{mach}$ is machine epsilon to prevent taking the $\log$ of $0$.
\par 
Once we did this, we formed the matrix $\Big( \mat{R^{T} R} + \mu \mat{I} \Big)$ where $\mu = 10^{mag}$.
We then used a simple while loop to iteratively solve the system discussed previously, and we set the stopping criterion to be when the relative $2$-norm of the current residual, defined as
\begin{align*}
	res_{k} & = \frac{||\vec{\hat{f}} - \mat{R} \, \vec{f}^{k}||_{2}}{||\vec{\hat{f}}||_{2}}
\end{align*}
falls below some previously specified tolerance $tol$.
\par 
As expected, the system did converge with the speed of the convergence dictated by the size of $\mu$.
If $\mu$ was much larger large relative to $\lambda_{min}$ $\mat{R^{T} R}$, the above system would take a large number of iterations until $res_{k}$ fell below a reasonable tolerance, say $10^{-10}$.
If $\mu$ was not much larger than $\lambda_{min}$, the above system would take a few iterations for the residual to fall below $10^{-10}$, but the conditioning of the system would be similar to $\mat{R^{T} R}$ (which could be very large).
\par 
However, the number of iterations grew very large even when our problem size was modest, so we scrapped the idea of iteratively solving this modified form of the normal equations.