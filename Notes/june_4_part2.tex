% !TEX root = main.tex

Today we wrote the code \texttt{spec\_diff.py}, a module that contains the function \texttt{spectral\_diff}.
This module takes as input a number of discretization points in space, optimized to be Chebyshev points, and optionally the range of those points. 
It returns the location of the discretization points, as well as the associated spectral differentiation matrix.
Its declaration and use is as follows:
$$\texttt{s, D\_N = spectral\_diff( N, a = -1, b = 1)}$$

This module is implemented through heavy use of numpy array indexing and broadcasting. 
This is done in an effort to limit the number of for-loops required, as the current code has none.
The diagonal of the matrix is constructed according to the numerically stable method described in Trefethen.
This method takes advantage of the fact that the sum of the columns in each row is 0.

Additionally, we begun work in the module \texttt{beuler\_advection.py}, which in its present state contains code to solve the following advection equation:
\begin{align*}
	&\hat{q}_{,t} + \hat{q}_{,s} = 0\\
	&\hat{q}(t=0, s) = \hat{q}_0(s) \quad s \in [-1, 1]
\end{align*}
The equation is discretized in space according to the above spectral method code, and is discretized in time with Backwards Euler. 
Presently, the code is incomplete, as solutions blow up near the edges of the domain.
It is suspected that the source of the error is mathematical, rather than programmatic. 
