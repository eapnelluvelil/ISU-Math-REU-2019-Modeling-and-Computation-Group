% !TEX root = main.tex

\subsection*{Time-stepping}
To explore time-stepping, we consider the advection equation:
\begin{align*}
	\hat{q}_{, t}(t, s) + \hat{q}_{, s}(s, t) & = 0, \, s \in [-1, 1] \\
	\hat{q}(t = 0, s) & = \hat{q}_{0}(s)
\end{align*}
If we discretize in space using the Chebyshev points (i.e. we consider the function values at the Chebyshev points on the interval $[-1, 1]$), we can use previous results to write that
\begin{align*}
	\hat{\vec{q}}_{, s} & \approx \mat{D_{N}} \, \hat{\vec{Q}}
\end{align*}
where $\mat{D_{N}}$ is the Chebyshev differentiation matrix corresponding to $(N + 1)$ Chebyshev points on the interval $[-1, 1]$ and $\hat{\vec{Q}}$ is the $(N+1) \times 1$ vector containing the approximate solution at the $(N + 1)$ Chebyshev points on the interval $[-1, 1]$.
\par 
Using the approximation to the partial derivative with respect to space, we are to able to obtain the semi-discrete form of the above PDE
\begin{align*}
	\frac{\partial}{\partial t} \hat{\vec{Q}} + \mat{D_{N}} \, \hat{\vec{Q}} & = \vec{0}
\end{align*}
which we can solve using the \underline{method of lines}.
\par 
To time-step, we can discretize the partial derivative with respect to time to obtain the fully discretized form of the PDE and pick a time-stepping method.
For example, we can use the \underline{forward Euler} method to time-step.
Forward Euler discretizes the partial derivative with respect to time in the following manner:
\begin{align*}
	\frac{\hat{\vec{Q}}^{n+1} - \hat{\vec{Q}}^{n}}{\triangle t} + \mat{D_{N}} \, \hat{\vec{Q}}^{n} & = \vec{0}
\end{align*}
where $\hat{\vec{Q}}^{n+1}$ is the approximate solution at the $(n+1)^{th}$ time step,  $\hat{\vec{Q}}^{n+1}$ is the approximate solution at the $n^{th}$ time-step, and $\triangle t$ is the time-step size. The forward Euler method can be easily derived.
\par 
We can rearrange the above equation as follows:
\begin{align*}
	\hat{\vec{Q}}^{n+1} & = \hat{\vec{Q}}^{n} - \triangle t \mat{D_{N}} \, \hat{\vec{Q}}^{n} \\
						& = \Big( \mat{I} - \triangle t  \mat{D_{N}} \Big) \hat{\vec{Q}}^{n}
\end{align*}
which allows us to compute the approximate solution at the Chebyshev points at the next time-step using the approximate solution at the current time-step.
Note that forward Euler is an \underline{explicit method}, and there is a dependence between the spatial and time-step size. 
If the time-step size is too large, as we compute the approximate solutions at successive time steps, the solutions will display numerical instability. 
We can use more stable explicit methods, such as the \underline{Runge-Kutta fourth order} method, but we still have to pay attention to the spatial and time step sizes.
\par 
We can also use \underline{implicit methods} to time-step.
The simplest implicit method is \underline{backward Euler}, and it discretizes the partial derivative with respect to time in the following manner:
\begin{align*}
	\frac{\hat{\vec{Q}}^{n+1} - \hat{\vec{Q}}^{n}}{\triangle t} + \mat{D_{N}} \, \hat{\vec{Q}}^{n+1} & = \vec{0}
\end{align*}
We can derive the backward Euler method by noting that we can put the PDE into the form
\begin{align*}
	\hat{q}_{, t}(t, s) & = -\hat{q}_{, s}(t, s)
\end{align*}
Since the above equation must hold for all time $t > t_{0}$ (where $t_0$ is the initial time), it must hold at the $(n+1)^{th}$ time step (denoted $t_{n+1}$).
We can use a backward difference approximation to approximate $\underline{\hat{q}}_{, t}$ at $t_{n+1}$, i.e.
\begin{align*}
	\vec{\hat{q}}_{, t}^{n+1} & \approx \frac{\hat{\vec{Q}}^{n+1} - \hat{\vec{Q}}^{n}}{\triangle t}
\end{align*}
and from previous results, we can approximate the partial derivative of $\hat{q}$ using the Chebyshev differentiation matrix i.e.
\begin{align*}
	\vec{\hat{q}}_{, s}^{n+1} & \approx \mat{D_{N}} \, \vec{\hat{Q}}^{n+1} 
\end{align*}
Substituting these approximations into the PDE, we get
\begin{align*}
	\frac{\hat{\vec{Q}}^{n+1} - \hat{\vec{Q}}^{n}}{\triangle t} & = -\mat{D_{N}} \, \vec{\hat{Q}}^{n+1} 
\end{align*}
After moving like terms to one side, we get
\begin{align*}
	\hat{\vec{Q}}^{n} & = \hat{\vec{Q}}^{n+1} + \triangle t \, \mat{D_{N}} \, \vec{\hat{Q}}^{n+1} \\
					  & = \Big( \mat{I} +  \triangle t \, \mat{D_{N}} \Big) \hat{\vec{Q}}^{n+1}
\end{align*}
We multiply the above equation by the inverse of the matrix (assuming the matrix is non-singular) on the right side to obtain
\begin{align*}
	\hat{\vec{Q}}^{n+1} & = \Big( \mat{I} +  \triangle t \, \mat{D_{N}} \Big)^{-1} \hat{\vec{Q}}^{n}
\end{align*}