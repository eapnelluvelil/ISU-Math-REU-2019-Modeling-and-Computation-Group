% !TEX root = main.tex

Suppose that the $xy$ coordinate system is the original coordinate system, and we have rotated the coordinate system by an angle $\omega$ to obtain the $sz$ coordinate system.
We can convert between the two coordinate systems using the following relationships:
\begin{align*}
	x(s, z; \omega) & = s \cos (\omega) - z \sin (\omega) \\
	y(s, z; \omega) & = s \sin(\omega) + z \cos (\omega)
\end{align*}
Once we pick $\omega$, we can measure $\hat{f}(s_{k}; \omega)$ for $k = 1, 2, 3, \hdots$ (for however many receiver locations there are).
We can then pick another $\omega$ and repeat the above process. 
\par 
Suppose $s$ is a continuous variable.
We denote $\hat{f}(s, \omega)$ to be the Radon transform of $f(x, y)$, and is defined by
\begin{align*}
	\hat{f}(s, \omega) & := \int_{-\infty}^{\infty} f(x, y) dz \\
					   & = \int_{-\infty}^{\infty} f\big( s \cos (\omega) - z \sin (\omega), s \sin(\omega) + z \cos (\omega) \big) dz
\end{align*}
where we have used the relationships between the $xy$ and $sz$ coordinate systems.
Note that the $z$ variable disappears because we are taking the line integral with respect to $z$ in the above definition. 
\par 
The \underline{imaging problem} (or inverse problem) is given $\hat{f}(s, \omega)$ (the Radon transform of $f(x, y)$), we want to recover $f(x, y)$.
\par 
Suppose we fix $\omega$.
Using the chain rule and the previous relationships between the $xy$ and $sz$ coordinate systems, we get that
\begin{align*}
	\frac{\partial}{\partial x} & = \frac{\partial s}{\partial x} \frac{\partial}{\partial s} + \frac{\partial z}{\partial x} \frac{\partial}{\partial z} \\
	\frac{\partial}{\partial y} & = \frac{\partial s}{\partial y} \frac{\partial}{\partial s} + \frac{\partial z}{\partial y} \frac{\partial}{\partial z}
\end{align*}
We can write the relationships between the $xy$ and $sz$ coordinate systems in matrix form:
\begin{align*}
	\begin{bmatrix}
		x \\
		y
	\end{bmatrix}
	& = 
	\begin{bmatrix}
		\cos (\omega) & - \sin (\omega) \\
		\sin (\omega) & \cos (\omega)
	\end{bmatrix}
	\begin{bmatrix}
		s \\
		z
	\end{bmatrix} \\
	& = C 
	\begin{bmatrix}
	s \\
	z
	\end{bmatrix}
\end{align*}
where $C = \begin{bmatrix} \cos (\omega) & - \sin (\omega) \\ \sin (\omega) & \cos (\omega)\end{bmatrix}$.
\par 
Multiplying the above equation on both sides by the inverse of $C$ (which we can easily verify is given by the following),
\begin{align*}
	C^{-1} & = 
	\begin{bmatrix}
		\cos (\omega)  & \sin(\omega) \\
		-\sin (\omega) & \cos(\omega)
	\end{bmatrix}
\end{align*}
we get that
\begin{align*}
	\begin{bmatrix}
	s \\
	z
	\end{bmatrix}
	& =
	\begin{bmatrix}
		\cos (\omega) & \sin (\omega) \\
		-\sin (\omega) & \cos (\omega)
	\end{bmatrix}
	\begin{bmatrix}
		x \\
		y
	\end{bmatrix}
\end{align*}
We see that
\begin{align*}
	\frac{\partial s}{\partial x} & = \cos (\omega) & \frac{\partial s}{\partial y} & = \sin (\omega) \\
	\frac{\partial z}{\partial x} & = -\sin (\omega) & \frac{\partial z}{\partial y} & = \cos (\omega)
\end{align*}
Thus
\begin{align*}
	\frac{\partial}{\partial x} & = \cos (\omega) \frac{\partial}{\partial s} -\sin (\omega) \frac{\partial}{\partial z} \\
	\frac{\partial}{\partial y} & = \sin (\omega) \frac{\partial}{\partial s} + \cos (\omega) \frac{\partial}{\partial z}
\end{align*}
We are interested in the Radon transform of $\frac{\partial f}{\partial x} := f_{, x}$.
By definition of the Radon transform from earlier, we have that $\widehat{f_{, x}}(s, w)$ (the Radon transform of $f_{, x}$) is defined as follows:
\begin{align*}
	\widehat{f_{, x}}(s, \omega) & = \int_{-\infty}^{\infty} f_{, x}(x, y) \, dz \\
							& = \int_{-\infty}^{\infty} f_{, x}(s \cos (\omega) - z \sin (\omega), s \sin (\omega) + z \cos (\omega)) \, dz \\
							& = \int_{-\infty}^{\infty} \Big( \cos (\omega) f_{, s} - \sin (\omega) f_{, z} \Big) \, dz \\
							& = \cos (\omega) \int_{-\infty}^{\infty} f_{, s} \, dz - \sin (\omega) \int_{-\infty}^{\infty} f_{, z} \, dz
\end{align*}
We can pull the partial derivative with respect to $s$ out of the first term in the above expression.
Furthermore, if we assume that $f$ has compact support, we get that $f(z = \infty) = 0$ and $f(z = -\infty) = 0$.
Thus,
\begin{align*}
	\int_{-\infty}^{\infty} \, f_{, z} dz & = f(z = \infty) - f(z = -\infty) \\
									   & = 0 - 0 \\
									   & = 0
\end{align*}
Thus, we get that
\begin{align*}
	\widehat{f_{, x}}(s, \omega) = \cos (\omega) \frac{\partial}{\partial s} \Bigg( \int_{-\infty}^{\infty} f \, dz \Bigg)
\end{align*}
From earlier, we know that $\hat{f}(s, \omega) = \int_{-\infty}^{\infty} f \, dz$, so we get that
\begin{align*}
	\widehat{f_{, x}}(s, \omega) & = \cos (\omega) \frac{\partial}{\partial s} \hat{f} (s, \omega) \\
								 & = \cos (\omega) \hat{f}_{, s} (s, \omega)
\end{align*}
where $\hat{f}_{, s} (s, \omega)$ denotes the partial derivative with respect to $s$ of the Radon transform of $f$.
In alternate notation,
\begin{align*}
	R(f_{, x}) (s, \omega) & = \cos (\omega) \frac{\partial}{\partial s} R(f) (s, \omega)
\end{align*}
We can compute the Radon transform of $\frac{\partial f}{\partial y} := f_{, y}$ in a similar manner.
\begin{align*}
	\widehat{f_{, y}}(s, \omega) & = \int_{-\infty}^{\infty} f_{, y} \, dz \\
								 & = \int_{-\infty}^{\infty} \Big( \sin(\omega) f_{, s} + \cos (\omega) f_{, z} \Big) \, dz \\
								 & = \sin (\omega) \frac{\partial}{\partial s} \int_{-\infty}^{\infty} f \, dz \\
								 & = \sin(\omega) \frac{\partial}{\partial s} \hat{f} (s, \omega) \\
								 & = \sin (\omega) \hat{f}_{, s} (s, \omega)
\end{align*}
In alternate notation,
\begin{align*}
	R(f_{, y}) (s, \omega) & = \sin (\omega) \frac{\partial}{\partial s} R(f) (s, \omega)
\end{align*}
We see that when we take Radon transforms of the partial derivatives of $f$, we end up dealing with only the partial derivatives of the Radon transform of $f$ with respect to $s$.