% !TEX root = main.tex

Combining the Randon Transform and hyperbolic PDEs: \\ 
We begin with our general form of the hyperbolic PDE which is a two-dimensional function with time
\begin{align*}
\vec{q}_{,t} + \mat{A}\,\vec{q}_{,x} + \mat{B}\,\vec{q}_{,y} = \vec{0}
\end{align*}
and then we begin to compute the Randon Transform on it. \\ 
Note: $q(t,x,y)$ will become $q(s,x)$ after the transformation. \\

\begin{align*}
R(\vec{q}_{,t}) = \frac{\partial}{\partial t} R(\vec{q}) \\
R(\mat{A}\,\vec{q}_{,x}) = \mat{A}R(\vec{q}_{,x}) = \mat{A}\cos(\omega)\frac{\partial}{\partial s}R(\vec{q}) \\
R(\mat{B}\,\vec{q}_{,y}) = \mat{B}R(\vec{q}_{,y}) = \mat{B}\sin(\omega)\frac{\partial}{\partial s}R(\vec{q}) \\
\end{align*}
We then combine these into the form:
\begin{align*}
\frac{\partial}{\partial t} R(\vec{q}) + \mat{A}\cos(\omega)\frac{\partial}{\partial s}R(\vec{q}) + \mat{B}\sin(\omega)\frac{\partial}{\partial s}R(\vec{q}) = \vec{0} \\
\widehat{\vec{q}}_{,t} + \widetilde{\mat{A}}(\omega)\widehat{\vec{q}}_{,s} = \vec{0} \\
\widetilde{\mat{A}}(\omega) = \cos(\omega)\mat{A} + \sin(\omega)\mat{B}
\end{align*}
The second equation is a collection of one-dimensional problems parameterized by $\omega$. \\

Initial Value Problems:
\begin{align*}
\vec{q}_{,t} + \mat{A}\,\vec{q}_{,x} + \mat{B}\,\vec{q}_{,y} = \vec{0} \\
q(t=0,x,y) = q_0(x,y)
\end{align*}

Process: \\
Step 1: Compute $R(q_0(x,y))$ \\
Step 2: Solve from $t=0$ to $t=T$
\begin{align*}
\widehat{\vec{q}}_{,t} + \widetilde{\mat{A}}(\omega)\widehat{\vec{q}}_{,s} = \vec{0} \\
\widehat{\vec{q}}(t=0,x,y) = R(q_0(x,y)
\end{align*}
Step 3: Compute $R^{-1}(\widehat{q}(T,s,w)$