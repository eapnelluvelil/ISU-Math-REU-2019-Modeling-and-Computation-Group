% !TEX root = main.tex

\subsection{Application}

The application of interest to us is Radiative Transfer, a model for subatomic particles (specifically photons and neutrons) propogating in a homogenous medium.
The principle governing equation is as follows:
\begin{align*}
	F_{,t} + \vec{\Omega} \cdot \vec{\nabla}F + \sigma F = \frac{\sigma}{4\pi}\int_{\mathbb{S}^2}F\,d\vec\Omega
\end{align*}
We want to solve for the distribution of particles
\begin{align*}
	F( t, \vec{x}, \vec{\Omega}) : \quad \R^+ \times \R^3 \times \mathbb{S}^2 \to \R
\end{align*}
where $t$ is time, $\vec{x}$ is a position, and $\vec{\Omega}$ are angular variables.

In many ways, this equation is relatively simple, consisting of two "types" of terms.
The first of these is $ \vec{\Omega} \cdot \vec{\nabla}F $, which determines the advective character of the particles.
This is essentially transport with velocity $\vec{\Omega}$ (Note that $||\vec{\Omega}|| = 1$, thus all particles have speed exactly equal to 1).

The other terms can be combined into the collision term 
\begin{align*}
	\sigma\left(\frac{\sigma}{4\pi}\int_{\mathbb{S}^2}F\,d\vec\Omega - F\right).
\end{align*}
As an explanation for this terms effect on $F$, consider the following simple ordinary differential equation:
\begin{align*}
	F_{,t} = \sigma( G - F ).
\end{align*}
The solution to this equation is $F(t) = G + ce^{-\sigma t}$, which approaches $G$ as $t$ approaches infinity.
Analogously, this term in our model causes $F$ to "approach" $\frac{\sigma}{4\pi}\int_{\mathbb{S}^2}F\,d\vec\Omega\,$ as $t\to \infty$.
Moreover, because $\frac{\sigma}{4\pi}\int_{\mathbb{S}^2}F\,d\vec\Omega\,$ is equivalent to the angular average of the particles, this term indicates that the collisions in the system removes its angular dependence.
In other words, the particles are uniform across angles, and form an isotropic state.

When arranged into the following form, the model becomes a linear transport equation:
\begin{align*}
	F_{,t} + \vec{\Omega} \cdot \vec{\nabla}F = \sigma \left(\frac{\sigma}{4\pi}\int_{\mathbb{S}^2}F\,d\vec\Omega - F\right)
\end{align*}

However, while the PDE can be expressed simply, it is far from trivial. 
In particular, it is still a distribution functino in as many as $ 1 + 5$ dimensions.
This makes it fall under the influence of the curse of dimensionality, in which computational cost increases exponentially with dimension.
Typically, a problem that can be solved in $O(N)$ in 1D, must be solved in $O(N^d)$ in $d$-dimensions.
As a result some simplifications must be done on this problem to make it solvable in a resonable computation time.