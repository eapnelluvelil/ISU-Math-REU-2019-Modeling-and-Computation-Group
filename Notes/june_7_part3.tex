% !TEX root = main.tex

% June 7, Part 3

\section*{Daily update on code}
Today, we worked on generalizing the one-dimensional wave equation code so that it would solve more than two advection equations with absorbing boundary conditions simultaneously.
We begin with the potentially coupled system of equations we obtain by transforming the wave equation
\begin{align}
	\vec{q}_{, t} + \mat{A} \, \vec{q}_{, x} & = \vec{0}
\end{align}
We assume that $\mat{A}$ is a symmetric $M \times M$ matrix with real entries and that we can take the spectral decomposition of $\mat{A}$ as follows
\begin{align}
	\mat{A} & = \mat{R} \, \mat{\Lambda} \, \mat{R^{-1}}
\end{align}
where $\mat{R}$ is the matrix whose columns are the eigenvectors of $\mat{A}$ and $\mat{\Lambda}$ is the diagonal matrix whose diagonal entries contain the eigenvalues of $\mat{A}$.
We can then use the above decomposition to write the original system of equations as
\begin{align}
	\vec{q}_{, t} + \mat{R} \, \mat{\Lambda} \, \mat{R^{-1}} \, \vec{q}_{, x} & = \vec{0}
\end{align}
We multiply the above equation through by $\mat{R^{-1}}$ to obtain
\begin{align}
	\mat{R^{-1}} \, \vec{q}_{, t} + \mat{\Lambda} \, \mat{R^{-1}} \, \vec{q}_{, x} & = \vec{0}
\end{align}
Since $\mat{R}$ (and $\mat{R^{-1}}$) are constant matrices, differentiation with respect to time or space will not affect these matrices, so can make the following substitution
\begin{align}
	\vec{w} & = \mat{R^{-1}} \, \vec{q}
\end{align}
We can now rewrite the equation as
\begin{align} 
	\vec{w}_{, t} + \mat{\Lambda} \, \vec{w}_{, x} & = \vec{0}
\end{align}
The $i^{th}$ equation in the above system reads as 
\begin{align}
	w_{i, t} + \lambda_{i} w_{i, x} & = 0
\end{align}
We can obtain the semi-discrete form of the above equation by replacing the partial derivative with respect to space with $\mat{D_{N}}$ (where $N + 1$ is the number of Chebyshev points we use) and discretizing the solution $w_{i}$
\begin{align}
	\vec{w_{i}}_{, t} + \lambda_{i} \mat{D_{N}} \, \vec{w_{i}} & = 0
\end{align}
Depending on the sign of $\lambda_{i}$, we need to modify $\mat{D_{N}}$ to include the absorbing boundary conditions i.e. if $\lambda_{i}$ is negative, we need to zero out the first column and row of $\mat{D_{N}}$ and vice versa. 
At this point, we can discretize the above equation with respect to time and apply backward Euler, the trapezoidal rule, or the fourth order extension of the trapezoidal rule to time-step.