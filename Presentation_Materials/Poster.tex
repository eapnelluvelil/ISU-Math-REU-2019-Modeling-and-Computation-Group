Title:													 										???
Solving Radiative Transfer Equations using the Radon Transform

Authors:																						???
Eappen Nelluvelil, Megan Oeltjenbruns, Dr. James Rossmainth, Jacob Spainhour, Christine Wiersma

Intro: 																						???
We set out to solve radiative transfer equations using the radon transform. We started out writing code to solve a hyperbolic PDE. Then we coded the spherical harmonics used to change the radiative transfer equations. After that, we coded the radon transform and found a way to create an operator to apply to a function. Next, we tried to find a way to perform the inverse radon transform. Finally, we worked on combining all of this code into an easy-to-manage group of files.

Methods: 																						???
1. Use Spherical Harmonics to change Radiative Transfer Equations
2. Perform Radon Transform
3. Use hyperbolic properties to solve
4. Perform Inverse Radon Transform

Results:  																						???
Here are our results.

Error:  																						???
We had varying degrees of error in the various parts of our project. At the end, the error that we get from beginning to end is ??.

Conclusion:  																					???
Our conclusion is.

Future Research:																				???
Our Inverse Radon Transform could be improved upon.
This could be coded in a different language.
This could be expanded for use with a finer mesh of points with more memory capacity.

Acknowledgements:  																			???
NSF Grant, ISU Math Department

Images:																						???