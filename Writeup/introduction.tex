% !TEX root = main.tex

%%%%%%%%%%%%%%%%%%%%%%%%%%%%%%%%%%%%%%%%%%%%%%%%%%%%%%%%%%%%%%%%%%
\section{Introduction}

%%%%%%%%%%%%%%%%%%%%%%%%%%%%%%%%%%%%%%%%%%%%%%%%%%%%%%%%%%%%%%%%%%
\subsection{Imaging}

The Radon transformation commonly seen alongside imaging problems.
Fundamentally, the imaging problem attempts to reconstruct a distribution from measurements taken at several different angles.
A familiar example of this is any kind of medical imaging device, such as an MRI machine.
The device collects measurements from several angles, which are used to reconstructed an accurate image.
When applied to a mathematical function, the Radon transform produces these measurements, or more accurately profiles, with integrals.
Likewise, the inverse Radon transform restores the original function from a collection of profiles spanning all angles.
Many properties of the Radon transform cause standard PDE solution methods to be simpler on the profiles than on the original domain.
For this paper specifically, we take advantage of the dimension-intertwining properties of the Radon transform to simplify solution methods of hyperbolic PDEs.

%%%%%%%%%%%%%%%%%%%%%%%%%%%%%%%%%%%%%%%%%%%%%%%%%%%%%%%%%%%%%%%%%%
\subsection{Previous Work}

Existing literature regarding this method is sparse.
Much of the theoretical groundwork is owed to the work of \textit{Dimensional Splitting of Hyperbolic Partial Differential Equations using the Radon Transform} by Donsub Rim.\cite{Rim:5}
Here, Rim was able to use a coarse approximation to the Radon transform to use large time stepping methods to solve the hyperbolic PDEs we are interested in.
Our method follows the same basic procedure, but uses a more accurate forward and inverse Radon transform to produce more numerically viable results.